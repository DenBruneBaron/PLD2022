\section{Parameter passing mechanisms}
\subsection{a)}
Since we're only asked about what happens in this general case, my answer to this
question is a more general answer. I'm not doing the actual calculations of transpose(q,q)
In this case, the function is given the same parameter of the matrix q, hence "transpose(q,q).
This means, that the parameters are stored in two difference local variables in the callee (transpose).
If these local variables have different values when the call returns, there can be an ambiguity in which value is copied back to the variable.
However in this case, I know that it is copied (written) back from left to right.
So the "a" parameter will be copied back first. Overwriting the original value of the variable.

\subsection{b)}
Call-by-Reference can introduce "aliasing". Meaning that if a function is given the same variable to/as two different reference parameters.
Those two parameters (variables in the callee) can refer to the same location. Meaning that updating one of the variable also updates the other one.
\\
Aliasing hinders optimisation.

\subsection{c)}
If a compiler fetches the content of a reference parameter into a register, and then write to another reference parameter. 
The programmer should make sure that the compiler re-fetch the first parameter, because the write might have affected this.
This would occur if the parameters are aliases of the same location.

essentially when using pass-by-reference parameters, if a function causes changes to the variables in the caller
these changes would be reflected in the callee, since original values would have been overwritten.

% \begin{minted}[linenos, bgcolor = bg, breaklines]{LISP}
%     ; return the sum of a list
%     (defin sum
%         (lambda (())    0
%             ((a . as)) ( +a (sum as))))
% \end{minted}