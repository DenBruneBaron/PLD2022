\section{A3.4}
\subsection{a)}
Root and mixed are two different functions.
Root creates a the that has a root at X with either zero, two or three children.
Mixed is a function that checks whether the tree created by the root function is
a mixed tree. And will possibly return a boolean value if the tree created by root
is or is not indeed a mixed tree.

\subsection{b)}
\begin{verbatim}
    root("gorilla",
        root("goat",
            root("duck",
                root("koala"), Root("manatee")),
            root("impala")
        )
        root("horse"), Root("ostrich"))
\end{verbatim}

\subsection{c)}
Creating a method called: leftmost
and implementing this in prolog:

I have the following.

\begin{minted}[linenos, bgcolor = bg, breaklines]{prolog}
leftmost(root(E),E).
leftmost(root(_, T1, _), E) :- leftmost(T1, E).
leftmost(root(_, T1, _, _), E) :- leftmost(T1, E).

mixed(root(X, T1, T2, T3)) :- string(X), mixed(T1), mixed(T2), mixed(T3).
mixed(root(X, T1, T2)) :- string(X), mixed(T1), mixed(T2).
mixed(root(X) :- string(X).
\end{minted}

Query for calling the prolog function
\begin{minted}[linenos, bgcolor = bg, breaklines]{prolog}
leftmost(root("gorilla", root("goat", root("duck", root("koala"), root("manatee)), root("impala")), root("horse"), root("ostrch)), "koala")
\end{minted}

\subsection{d)}
