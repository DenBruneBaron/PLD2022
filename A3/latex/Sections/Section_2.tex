\section{A3.2}
\subsection{a)}

\textbf{Advantages}
\begin{itemize}
    \item Extensive control over the program and which parts of the program
    that should be able to use which modules. Can change the export statement in descendant classes.
    \item Security. Having the prosibility to restrict certain parts of the program
    from using specific moduels. (This looks very similar to public, private, protected as we see in Java. Only
    at a slighy more general level in Eiffel.)
\end{itemize}

\noindent\textbf{Disadvantages}
\begin{itemize}
    \item Risk of making to many internal functions unusable to other parts of the
    program. And or making the readability of the different exports a hassle to understand.
    \item 
\end{itemize}

\subsection{b)}

\textbf{Advantages}
\begin{itemize}
    \item Since every program in Modula-2 is composed of moduels, code can easily be resused.
    \item Encapsulation. This gives the programmer the option to restrict visibility of cartain
    parts of the code e.g. subprograms or data structures to other parts of a program.
    \item The possibility to only import specific methods from a module.
\end{itemize}

\noindent\textbf{Disadvantages}
\begin{itemize}
    \item Importing a whole module only to use a couple of methods, may take up file size.
    \item Assuming that modula-2 does not import all methods by default. The programmer
    would need to specify all modules in case a module contains many functions.
    \item The programmer must keep track of both input and output of the specific module.
\end{itemize}