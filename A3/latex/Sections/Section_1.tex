\section{A3.1}

\subsection{a)}
The compiler rejects the program with a compile error, since the method
"bingoString()" isn't defined for the generic type T.
\\
\\
This is the error message I get when I try to compile the program
\begin{verbatim}
Bingo.java:9: error: cannot find symbol
    System.out.println(t.bingoString());
                        ^
symbol: method bingoString()
location: variable t of type T
where T is a type-variable:
T extends Object declared in class Bingo
1 error
\end{verbatim}

\subsection{b)}
The following code should compile. But throw a NullPointerException. Otherwise, since we don't
really know what BingoString is supposed to do in this specific code snippet, a runtime error could
be cause by a wrong cast.

\begin{minted}[linenos, bgcolor = bg, breaklines]{java}
abstract class myAbstractClass {
    public abstract String bingoString();
}

class Bingo<T extends myAbstractClass> {

    public void dingo(T t) {
        System.out.println(t.bingoString());
    }


    public static void main(String[] args) {
        Bingo<myAbstractClass> myObj = new Bingo<>();
        myObj.dingo(null);
    }
}
\end{minted}

\subsection{c)}
Reusing the same piece of code, as in 1b only with a minor tweaks. This code will compile but do nothing.
\begin{minted}[linenos, bgcolor = bg, breaklines]{java}
abstract class myAbstractClass {
    public abstract String bingoString();
}

class Bingo<T extends myAbstractClass> {

    public void dingo(T t) {
        System.out.println(t.bingoString());
    }

    public static void main(String[] args) {
        
    }
}
\end{minted}

\subsection{d)}
Since we assume that the code will compile, this should lead to a run-time error.
since we can't set a subclass equal to the value of a superclass in java.

\subsection{e)}


