\section{PLD LISP}
\subsection{a) - PLD-LISP}
Reading through the file \emph{listfunctions.le} I've reused the code
for returning the length of a list. I've restructured it to be able
to return the sum of a list instead.

\begin{minted}[linenos, bgcolor = bg, breaklines]{LISP}
    ; return the sum of a list
    (defin sum
        (lambda (())    0
            ((a . as)) ( +a (sum as))))
\end{minted}

\subsection{b) - Python}
\begin{minted}[linenos, bgcolor = bg, breaklines]{python}
    # return the sum of a list
    def sum_of_list(lst):
        sum = 0
        for val in lst:
            total = total + val
        return total
\end{minted}

I've chosen to implement the same type of function in python since I know
python fairly well.
Comparing the size of the two programs they are fairly similarly, with the python
program being a couple of extra lines bigger compared to the PLD-LISP. This could
possible add up if a person were to implement a bigger program.
\\
Personally I know python better and I think it's easier to read that the PLD-LISP.
\\
PLD-LISP also has a numerous ammount of parantheses that can really mess with the 
programmer if you're not careful.
\\
As for effort to program the two functions, I personally found the python program
easier to make. Again, this is because I know python better than PLD-LISP, so in all
fairness I don't think this is a very representable example of the question regarding
"Syntaxes for size, readability and effor to program".
